\documentclass[oldfontcommands]{memoir} 
\usepackage{fullpage}
\usepackage{url}
\usepackage{amsmath}
\usepackage{graphicx}
\usepackage{fancyvrb}
\usepackage{color}
\usepackage{listings}
\usepackage{utopia}
\lstset{fancyvrb=true}
\lstset{ 
  basicstyle=\small\tt,
  keywordstyle=\color{blue},
  identifierstyle=,
  commentstyle=\color{green},
  stringstyle=\color{red},
  showstringspaces=false,
  tabsize=3,
  numbers=left,
  captionpos=b,
  numberstyle=\tiny
  %stepnumber=4
  }

% TODO subfigures using subfig

\let\oldhat\hat
\renewcommand{\vec}[1]{\mathbf{#1}}
\renewcommand{\hat}[1]{\oldhat{\vec{#1}}}

\newcommand{\explain}[1]{\paragraph{Explain:} \emph{#1}}


\title{CMSC 23710 Project 3: Derivatives and Isocontours}
\author{Damon Wang}
\date{2 November 2010}

\setcounter{secnumdepth}{2}

\begin{document}\tightlists

\maketitle

\section{Aspect ratios}

\begin{figure}
  \small
  \lstinputlisting[language=XML]{../svg/teddy.svg}
  \lstinputlisting[language=XML]{../svg/feeth.svg}
  \normalsize\caption{SVG to correct the aspect ratios on \texttt{teddy.png} and \texttt{feeth.png}}
\end{figure}

\section{Derivatives}

\explain{Knowing that the rings here are circular, how are you
able to assess the correctness of your result?}

Since the rings are circular, the magnitude of the gradient should be radially
symmetric and the plot of $\frac{\partial}{\partial x}$ should match that of
$\frac{\partial}{\partial y}$ up to a $90^\circ$ rotation. Just by visual
inspection, my results seem to meet both conditions.

\explain{In the CT foot data, what properties or features are made more visible
by either of these derivative images?}

The gradient visualisation makes edges more visible. Since X-ray opacity roughly
correlates with density, transitions between materials appear as sudden changes
in opacity. Right at the transition, the gradient has large magnitude, so the
foot and then the bones are outlined in white.

\end{document}


% vim: tw=80
